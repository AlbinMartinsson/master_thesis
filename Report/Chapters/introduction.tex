%Intro to intro
This thesis will examine the opportunity and possibility to connect an embedded IoT device to a local Eclipse Arrowhead framework cloud.
This project will use the STM32 B-L4S5I-IOT01A IoT discovery node as a development board, running the Mbed-OS 6. 
\section{Background}
%Setting
According to Artemis-IA, the world is on the verge of a new industrial revolution, moving towards a more decentralized and software-oriented means of production.\cite{Artemis2021}
This fourth new, and in some sense planned, industrial revolution is called Industry 4.0 according to  Lasi.\cite{Lasi2014} 
Incorporating System of Systems, Cyber-Physical Systems, and embedded software technologies will form the backbone and inherent part of every value chain. Then, this new industrial revolution is complete Artemis-IA means.\cite{Artemis2021}

%Technologies
Cyber-Physical systems act as a bridge between the data-rich cybernetic world and the technology-rich physical world Artemis-IA means.
Artemis-IA adds that a differentiating factor between traditional embedded systems and Cyber-Physical systems is their scale, where traditional embedded systems have a more limited scale.
Cyber-Physical system, on the other hand, has a much larger scale, including interconnected embedded systems, human-, and socio-technological systems as well Artemis-IA adds.\cite{Artemis2021}

%Economic incentives.   
For Europe to be able to compete with the rest of the world's larger economies with initiatives, for instance, China's Made in China 2025, Europe needs to invest in the technologies mentioned above Artemis-IA adds.
A shift away from proprietary solutions towards collaborative solutions is also needed means Artemis-IA in their report Embedded intelligence.\cite{Artemis2021} 
\section{Motivation}
%Solve problem presented in the background.
The need for a European incentive promoting Industry 4.0 is clear. 
According to their website, the Arrowhead Tools project aims for digitalization and automation solutions for the European industries.\cite{AT2021}
The Arrowhead Tools project uses the open-source Eclipse Arrowhead framework, further contributing to the collaborative solutions needed for the European economy defined in the previous section.

%What is there now?
The Eclipse Arrowhead framework contains many examples in various languages and technologies, mainly java. The language the framework is developed in. 
Python, C\#, and C++ have client libraries and example code developed for them, which can be found on the project's GitHub page.\cite{AC2021} 

%What is needed?
However, there is no client library or code example for a specific piece of hardware to connect to a local Arrowhead cloud fast and easy, showcasing the capabilities of this project. 
A project with the clear intent to showcase both the capabilities and possibilities of Cyber-Physical systems and the Eclipse Arrowhead framework is therefore needed.

%Conclussion to motivation
This thesis will examine the possibilities of having a ready-made example to compile and run on a specific hardware platform that connects a local Arrowhead cloud as a proof of concept. 
Much like the 'Getting started with' examples from Amazon Web Services and Microsoft Azure.\cite{Guide2020,AZURE2021}
\section{Problem definition}
%This is good, if deleting the other paragraph expand this.
This project aims to investigate the possibilities, benefits, and limitations of using the Eclipse Arrowhead framework on embedded devices in contrast to commercially available solutions such
as Amazon's Amazon Web Services and Microsoft Azure. 
%Maybe remove, how to measure?? 
\section{Equality and ethics}
%Expand
The ability to own and control your data is becoming rarer and rarer these days, with giant corporations establishing their cloud services.
As a consumer always takes a risk when pushing sensitive data to a cloud owned by someone else, the right to own your data should not have to be infringed upon. 
The Eclipse Arrowhead framework and the use of local clouds move the storage of your data from giant corporations to your own.
\section{Sustainability}
%Expand
The use of small embedded devices instead of monolithic machines used by the industry today provides a much-needed decrease in energy consumption for larger industries.
On a greater scale, the use of IoT devices would also enable preventive maintenance of components, reducing both the cost and materials required for maintenance later on.
\section{Delimitations}
\subsection{Security}
%Expand
This thesis does not cover a solution to the numerous security risks and issues associated with IoT devices. 
\subsection{Core systems}
%Expand
This thesis will also only cover the three core systems of the Eclipse Arrowhead framework, which are the service registry, authorization, and orchestrator. 
The STM32 B-L4S5I-IOT01A IoT discovery node will not host the Arrowhead framework on the board itself since the Arrowhead framework is too resource-heavy for such a small device.
Instead, the board will connect to a local Arrowhead cloud hosted by another computer in the same network. 
\subsection{Intercloud connection}
%Expand
It will also only cover connection within the same local Arrowhead cloud, intracloud, instead of using multiple clouds, intercloud.
Intercloud connection requires two more Arrowhead systems, gateway and gatekeeper, to operate and the configuration of those systems is beyond the scope of this thesis.
\section{Thesis structure}
In chapter 2, related work is presented, a literature review of IoT, Industry 4.0, security and the Eclipse Arrowhead framework is conducted. 
Chapter 3 theory is covered, describing what technologies and scientific methods were used in this thesis.
Chapter 4 covers implementation, describing how the different systems used in this thesis are designed from a software engineering perspective.
In chapter 5, an evaluation of the experiment conducted will be performed. 
Chapter 7 presents the conclusion of the work done in this thesis. The chapter also describes how to investigate further the questions raised in this thesis. 
In chapter 8, there is a list of references used in this thesis.
