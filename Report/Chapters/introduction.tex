\section{Background}
The number of devices connected to the internet has risen from 7 billion in 2018 to an estimated 35 billion in 2021 according to and that number is only going to increase
according to security today.[1] %https://securitytoday.com/Articles/2020/01/13/The-IoT-Rundown-for-2020.aspx?Page=2%. 
\section{Motivation}
With the numbers of devices connected to the internet, IOT-devices 
from now on, rising to an estimated 38.6 billion devices world wide
by 2025 the need to enable communication between those devices have
never been bigger.
\section{Problem definition [OLD]}

%%%%%%%%%%%%%%%%%%%%% OLD ONE %%%%%%%%%%%%%%%%%%%%%%%%%%%%%%%%%%%%%%%%%%
This project aims to investigate the difference between either 
using a central broker or using peer to peer to handle the comunication 
between devices in terms of latency, energy consumption, robustness and
security. 

This project also aims to investigate the possibilities and benefits of 
late binding, i.e. a name is associated
with a particular operation or object at runtime instead of during 
compilation, in combination with hardware generated certificates  
in order to mitigate some of the security  issues associated with 
IoT devices.

\section{Problem definition [NEW]}
This project aims to investigate the possibilities, benefits and 
limitations of using Arrowhead Framework on embedded devices
in contrast to commercially available solutions such 
as Amazons Amazon Web Services, AWS 
from now on, and Microsofts Azure.  
This project also aims to investigate the difference between either 
using a central broker, the MQTT protocol used by AWS and Azure or using peer to peer, 
HTTP protocol used by the Arrowhead framework to handle the comunication between devices in terms of latency, 
energy consumption and
security.

\section{Equality and ethics}
Equality and ethics are learning objectives for the program and should be reflected upon in the thesis if applicable. This section can be omitted if not at all relevant to the problem definition, but in many cases the thesis topic touches upon these topics even if it is outside the scope of the work itself and in such cases a single paragraph may be sufficient to cover the reflection.
\section{Sustainability}
Similar to the equality and ethics section this is one of the learning objectives for the program and should be reflected upon in the thesis if applicable. For a quick overview of what is considered to be included in sustainability you can see the united nations list of sustainability goals: https://www.un.org/sustainabledevelopment/sustainable-development-goals/
\section{Delimitations}
Describe what is not covered in the thesis. Things you realize may have to be addressed to create a complete solution, but that would be too much work, or that may simply be out of the scope of your scientific area.
\section{Thesis structure}
Describe how the rest of your thesis is organized. (e.g. In section 2 we discuss, in section 3 there is a... etc.). This is really just to help guide the reader to where different parts of your work can be found.
