%Intro to intro
This thesis will take a look at the opportunity and possibility to connect your embedded IoT devices to a local arrowhead framework cloud.
This project will use the STM32 B-L4S5I-IOT01A IoT discovery node as a development board, running the Mbed-OS 6. 
\section{Background}
%What is there now?
The Arrowhead framework contains a lot of examples in an array of different languages and technologies mainly java, the language the framework is developed in. 
Python, C\#, and C++ have client libraries and example code developed for them, which can be found on the project's GitHub page.\cite{Github2021} 
%What is needed?
However, there is no client library or code example for a specific piece of hardware to make the connection to a local arrowhead cloud fast and easy. 
This thesis will examine the possibilities of having a ready-made example to compile and run that enables connection to a local arrowhead cloud as a proof of concept. 
Much like the 'Hello world' examples from Amazon Web Services and Microsoft Azure. 

\section{Motivation}
With the numbers of devices connected to the Internet, IoT-devices from now on, rising to an estimated 38.6 billion devices worldwide by 2025 the need to enable communication between those devices has never been bigger.
\section{Problem definition}
%This is good, if deleting the other paragraph expand this.
This project aims to investigate the possibilities, benefits, and limitations of using the Eclipse Arrowhead framework on embedded devices in contrast to commercially available solutions such
as Amazon's Amazon Web Services and Microsoft Azure. 
%Maybe remove, how to measure?? 
The way to measure the difference between either 
using a central broker, the MQTT protocol used by AWS and Azure or using peer to peer, 
The HTTP protocol used by the Arrowhead framework to handle the communication between devices in terms of latency, 
energy consumption and
security.
\section{Equality and ethics}
The ability to own and control your data is becoming rarer and rarer these days with giant corporations establishing their cloud services.
You as a consumer always take a risk when pushing sensitive data to a cloud owned by someone else, the right to own your data should not have to be infringed upon. 
The arrowhead framework and the use of local clouds move the storage of your data from giant corporations to your own.
\section{Sustainability}
The use of small embedded devices instead of monolithic machines used by the industry today provides a much-needed decrease in energy consumption for larger industries.
The use of IoT devices would also on a greater scale enable preventive maintenance of components, reducing both the cost and materials required for maintenance later on.
\section{Delimitations}
\subsection{Security}
%Expand
This thesis does not cover a solution to the numerous security risks and issues associated with IoT devices. 
\subsection{Core systems}
%Expand
This thesis will also only cover the three core systems of the Eclipse Arrowhead framework, which are the service registry, authorization, and orchestrator. 
The STM32 B-L4S5I-IOT01A IoT discovery node will not host the Arrowhead framework on the board itself, since the Arrowhead framework is too resource-heavy for such a small device.
The board will instead connect to a local Arrowhead cloud hosted by another computer in the same network. 
\subsection{Intracloud connection}
%Expand
It will also only cover connection within the same local arrowhead cloud, intracloud, as opposed to using multiple clouds, intercloud.
Intercloud connection requires two more Arrowhead systems, gateway, and gatekeeper, to operate and the configuration of those systems is beyond the scope of this thesis.
\section{Thesis structure}
In chapter 2 related work is presented, a literature review of IoT, Industry 4.0, security, and the Eclipse Arrowhead framework is conducted. 

In chapter 3 theory is covered, describing what scientific methods were used in this thesis. 

Chapter 4 covers implementation, describing how the different systems used in this thesis uses are designed from a software engineering perspective.

In chapter 5 an evaluation of the experiment conducted will be performed. 

Chapter 7 presents the conclusion of the work done in this thesis. The chapter also describes how to further investigate the questions raised in this thesis. 

In chapter 8  there is a list of references used in this thesis.
