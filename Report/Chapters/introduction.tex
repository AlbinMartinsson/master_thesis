\section{Background}
The number of devices connected to the Internet has risen from 7 billions in 2018 to an estimated 35 billions in 2021 according to Security Today and that number is only going to increase according to a survey done by security today. \cite{ST2020}
\section{Motivation}
With the numbers of devices connected to the Internet, IOT-devices from now on, rising to an estimated 38.6 billion devices world wide by 2025 the need to enable communication between those devices have never been bigger.
\section{Problem definition}
This project aims to investigate the possibilities, benefits and limitations of using Eclipse Arrowhead framework on embedded devices in contrast to commercially available solutions such
as Amazons Amazon Web Services, AWS 
from now on, and Microsofts Azure. 
 
The way to measure the difference between either 
using a central broker, the MQTT protocol used by AWS and Azure or using peer to peer, 
HTTP protocol used by the Arrowhead framework to handle the communication between devices in terms of latency, 
energy consumption and
security.
\section{Equality and ethics}
The ability to own and control your data are becoming more and more rare these days with giant corporations establishing their own cloud services.
You as a consumer always takes a risk when pushing sensitive data the a cloud owned by someone else, the right to own your data should not have to be infringed upon. 
\section{Sustainability}
The use of small embeded devices instead of monolithic machines used by the industry today provides a much needed decrease in energy consumption.
\section{Delimitations}
This thesis will not cover a solution to the numerous secruity risks and issues associated with IoT-devices. 
This thesis will also only cover the three core systems of the Eclipse Arrowhead framework, which are the service registry, authorization and orchestrator. 
\section{Thesis structure}
In chapter 2 related work is presented, a literature review of IoT, Industry 4.0, security and the Eclipse Arrowhead framework is conducted. 

In chapter 3 theory is covered, describing what scientific methods where used in this thesis. 

Chapter 4 covers implementation, describing how the different systems used in this thesis uses are designed from a software engineering perspective.

In chapter 5 an evalution of the experimet conducted will be performed. 

Chapter 7 presents the conclusion of the work done in this thesis. The chapter also describes how to further investigate the questions raised in this thesis. 

In chapter 8  there is a list of references used in this thesis.
