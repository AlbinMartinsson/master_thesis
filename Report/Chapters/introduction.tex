\section{Background}
What has been done already, and why does this work need to be solved. Typically if the masters project is issued by a company it is because the company already has one part, but wish to improve it or extend it somehow. Describe the background to your work.
\section{Motivation}
Why is your work important? Typically this section relates strongly to the problems that you define in 1.3. The motivations can be societal, economical, or other things, but try and look past the task at hand and find out what impact your results will have.
\section{Problem definition}



This is really the section you should start writing first, even if it's located as 1.3. What scientific problems are you solving in your thesis. The reason why you should start defining these is that they define what you should write in all the other sections. The introduction gives an overview of the area of your problems. The background offers the background to why you are working with your problems. The motivation gives a motivation to why you are solving the problems. The related work described work which relates to your problems. The implementation describes how you have implemented a solution to your problems. The evaluation evaluates whether you have actually solved your problems. The discussion discusses each individual problem, how you addressed it, alternative solutions and shortcomings, etc. The conclusions and future work describes the final outcome of how you solved your problems and what is left to do.
\\ \\
If you write your problems first it is so much easier to write all the other sections because you already know what the focus of each section should be. It prevents a lot of the writers block that people often suffer from.
\section{Equality and ethics}
Equality and ethics are learning objectives for the program and should be reflected upon in the thesis if applicable. This section can be omitted if not at all relevant to the problem definition, but in many cases the thesis topic touches upon these topics even if it is outside the scope of the work itself and in such cases a single paragraph may be sufficient to cover the reflection.
\section{Sustainability}
Similar to the equality and ethics section this is one of the learning objectives for the program and should be reflected upon in the thesis if applicable. For a quick overview of what is considered to be included in sustainability you can see the united nations list of sustainability goals: https://www.un.org/sustainabledevelopment/sustainable-development-goals/
\section{Delimitations}
Describe what is not covered in the thesis. Things you realize may have to be addressed to create a complete solution, but that would be too much work, or that may simply be out of the scope of your scientific area.
\section{Thesis structure}
Describe how the rest of your thesis is organized. (e.g. In section 2 we discuss, in section 3 there is a... etc.). This is really just to help guide the reader to where different parts of your work can be found.
