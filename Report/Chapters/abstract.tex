%Problem statement
This thesis will investigate the possibility of connecting an embedded device, STM32 B-L4S5I-IOT01A IoT discovery node, to a local Eclipse Arrowhead framework cloud.
This thesis will also compare the benefits and limitations of using the Eclipse Arrowhead framework to commercially available solutions such
as Amazon's Amazon Web Services and Microsoft Azure.

%Intro to motivation.
The world is on the verge of a new industrial revolution, often referred to as Industy 4.0, moving towards a more decentralized and software-oriented means of production.
Incorporating System of Systems, Cyber-Physical Systems, and embedded software technologies will form the backbone and be an inherent part of every value chain.

%Motivation
The Eclipse Arrowhead framework contains many examples in various languages and technologies but lacks an example of a specific piece of hardware connecting to a local Eclipse Arrowhead cloud.
Therefore, a project with the clear intent to showcase both the capabilities and possibilities of Cyber-Physical systems and the Eclipse Arrowhead framework is needed.

%Method.
The system consists of three major parts: the stm32 board, a Python flask app, and the Eclipse Arrowhead framework.
The main objective of the Eclipse Arrowhead framework is to connect the consumer and the provider in a safe and structured way.
The provider is built with C/C++ using ARMs' mbed os. 

%Tests.
The response time of the different frameworks, Eclipse Arrowhead framework and Amazon Web Services, was measured.
We made a thousand attempts to form an adequate basis for an average response time. 
In addition to presenting the average response time, we will calculate the maximum and minimum response times to understand the different frameworks' performance further. 

%Result.
The thesis also examined the benefits of using the Eclipse Arrowhead framework compared to its competitors Amazon Web Services and Microsoft Azure.
The thesis showed some benefits in terms of response time when running a local cloud instead of using a remote service such as Amazon Web Services, a 17.5 decrease in average response time was recorded.
Maximum and minimum response times decreased by 1.9 and 134 times, respectively.  