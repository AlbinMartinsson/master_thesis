This chapter summarizes the thesis by outlining what was accomplished and what remains to be done.  
Section 7.1 contains the conclusion, and a description of work to be done can be found in section 7.2.
\section{Conclusion}
This thesis examined the possibility of using an Eclipse Arrowhead framework local cloud on an embedded device, namely the STM32 B-L4S5I-IOT01A board. 
It was possible to implement functionality for the Eclipse Arrowhead framework on the STM32 B-L4S5I-IOT01A board, having the board send temperature readings to a computer within a local network.

Another goal of this thesis was to provide an easy-to-run example of the Eclipse Arrowhead framework that could be compiled by anyone wanting to give the framework a try, a feature that is missing right now for embedded devices.
The implementation was done with usability in mind, making it easy for users to try the code.
This was achieved with the Mbed online compiler's code importation and minimum setup development environment, allowing users to try out the Eclipse Arrowhead framework on the STM32 B-L4S5I-IOT01A board.

The thesis also examined the benefits of using the Eclipse Arrowhead framework compared to its competitors Amazon Web Services and Microsoft Azure.
The thesis showed some benefits in terms of response time when running a local cloud instead of using a remote service such as Amazon Web Services, a 17.5 decrease in average response time was recorded.
Maximum and minimum response times decreased by 1.9 and 134 times, respectively.  

\section{Future work}
%Intro.
The research presented in this thesis can take many different directions moving forward.
Two main issues were raised during this thesis, security and having the STM32 B-L4S5I-IOT01A board act as a server.
These two issues will be addressed in the following sections.
\subsection{Security}
%Security.
The security issues raised in both the related work and the previous chapter need to be addressed for this implementation ever to be used by the industry in production. 
Several attempts were made at implementing HTTPS; the Mbed-HTTP library has support for HTTPS, and STM cube IDE has support for wolfSSL. 

The problem that has to be solved before this could be implemented and used in the Eclipse Arrowhead framework is how certificates are handled in languages other than Java.
Both Mbed and STM cube has examples using HTTPS that works, and getting started with Amazon Web Services uses HTTPS with a user-generated certificate.
One of the main issues is that the certificates are self-signed, meaning no trusted certificate authority has signed them. 
This proved to be the main obstacle for implementing HTTPS using C. None of the libraries, Mbed-HTTP or wolfSSL, could trust the certificate from the Eclipse Arrowhead framework.

Further research on using the certificates generated by the Eclipse Arrowhead framework on embedded devices needs to be conducted.
One area of research could be to move away from the .pk12 format, generally used by java applications, and include more support for the .pem format used by C and many other languages.
Another area of research needed is lightweight cryptography and possible ways to move away from the idea that an IoT device has its certificate.
A concept that quickly becomes unbearable when dealing with thousands of devices in one network.

\subsection{Server implementation}
%Server implementation.
The need to have the STM32 B-L4S5I-IOT01A board as a passive component reacting to requests is also an issue that has to be resolved before the industry will use the results from this thesis in production.
Both the Mbed online compiler and the STM Cube IDE have a working example of an HTTP server that can request the temperature data from a generated webpage.
Those examples use pure HTTP requests and responses, leading to very verbose and challenging messages to parse. 
Future research that promotes the same usability as Mbed-HTTP and the ability to respond to the request would greatly benefit the Eclipse Arrowhead framework.  

A server implementation on the STM32 B-L4S5I-IOT01A board could also have great educational potential by using it in courses for young adults or aspiring engineers.
With the number of IoT devices connecting to the internet only increasing, understanding connected embedded devices is crucial for future engineers. 
Introducing concepts like IoT and embedded system programming early in an engineering degree and real-life examples could enhance knowledge and spike interest for those subjects, making aspiring engineers ready for the future.  
