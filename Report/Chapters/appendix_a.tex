\section*{Prerequisites}

\subsection*{Arrowhead local cloud in insecure mode}
You need a local arrowhead cloud with all core services running insecure mode in order to run this example. 
For this example I used the compile code option to install arrowhead. To disable secure make sure to edit the ssl.enabled = true to false in all the core systems, i.e. service registry, authorization and orchestrator, application.properties file. 
That file is located in the target folder of each individual core system. 

Go to eclipse arrowheads github \url{https://github.com/eclipse-arrowhead/core-java-spring} for information on that.

\subsection*{ST-Link drivers}
You need to install the ST-link drivers from \url{https://www.st.com/content/st_com/en/products/development-tools/software-development-tools/stm32-software-development-tools/stm32-utilities/stsw-link009.html}.
In order to download you have to have an ST account. 
Download the file, unzip and install the version approriate to your system. 
To check that it works plug in your st-device and make sure it appears as a unit on your computer.
\subsection*{Tera term}
To view the output this example you need to have tera term installed. Tera term can be downloaded from \url{https://osdn.net/projects/ttssh2/releases/}.
Once there download and install the latest version. 
\subsubsection*{Tera tern setup}
When you start tera term you need to change a couple of settings for the terminal so that the output presents itself in a nice way. In the terminal settings change new-line receive to AUTO and transmit to LF. 
\subsection*{MBED online compiler.}
Create an account on \url{https://ide.mbed.com/compiler/} and import the project from github.
Make sure that when selecting device to compile to you select B-L475E-IOT01A, even if you have B-L4S5I-IOT01A otherwise the example will not compile.
In order to import this project into your MBED online compiler simply following this link \url{https://os.mbed.com/compiler/#import:https://github.com/AlbinMartinsson/arrowhead_stm32_demo}. 

\section*{Configuration} 
\subsection*{\text{mbed\_app.json}}
In this file you have to change the wifi credentials, add your SSID and password. 
\subsection*{\text{main.cpp}} 
Here you have to change the address on all calls from localhost to the IP-address of your local arrowhead cloud. You also have to change the address of your provider and consumer body to match this. The provider address should be the same as your arrowhead cloud and the consumer should be the address of your stm32 board. Take note of which port you define to the provider as you will use it in your flask app.
\subsection*{\text{flaskapp.py}}
Make sure that the port matches the port of the provier system in main.cpp and that the address is set to 0.0.0.0 to allow external communication. 
\section*{Compilation and running the demo}
\subsection*{flask app}
Start the flask app with the run configurations specified in the previous section.
\subsection*{mbed compiler}
Do the changes in the previous section and hit ctrl+b to compile the project. This will download a .bin file. To run that .bin file on your board simply drag it to the board in the explorer. As if you were putting a file on a flash drive.
\subsection*{Expected output}
What this demo does is:
\begin{itemize}
    \item Register a provider system to the service registry.
    \item Register a consumer system to the service registry.
    \item Register a service to the service registry.
    \item Creates an intra cloud authorization rule.
    \item Creates an orchestrator store entry.
    \item Uses the id of the consumer to find an appropriate provider.
    \item Based on the response from the orchestration gets the url to the provider and sends the temperature as a json to it. 
\end{itemize}