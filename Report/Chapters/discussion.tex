This thesis aimed to investigate whether it was possible to use an embedded device in conjunction with the Eclipse Arrowhead framework.
If that was possible, this thesis also aimed to show the advantages and limitations of using the Eclipse Arrowhead framework on embedded devices.
Another goal of this thesis was to show that using the Arrowhead framework on embedded devices provides a ready-to-make example for people to try out the framework.
This was done with the end goal to fill a void in the Arrowhead project GitHub with a complete example ready to compile and using an online compiler.

\section{Choice of development environment}
The IDE, integrated development environment, chosen for this thesis stood between the Mbeb online compiler and the STM CUBE IDE. 
The STM cube IDE requires setup on the local machine, installing C and Cmake, but provides shorter compilation time.
The Mbed online compiler requires no setup on the local machine but takes much longer to save and compile the code.

Both compilation time and an easy setup are favorable attributes for an IDE, and it all depends on the end goal.
If the goal were to try out multiple different iterations of the same source code, then STM cube IDE would be the best choice.
On the other hand, if the goal was to provide an example of a framework or concept and get the users started as quickly and painless as possible, the Mbed online compiler is the obvious choice.
This thesis uses the latter because it fits the project's goals better.  


Both IDEs provide ready to run example that features different aspects of the development boards, and both contain an example of connecting the board to the internet.
The Mbed online compiler, in conjunction with arm Mbed OS, provides an easy and intuitive way to get started on desired projects.
The Mbed online compiler has its drawbacks as well. 
The Mbed compiler pushes the users into using libraries integrated into the Mbed OS 6.
It is possible to use outside libraries but more often than not, resulting in a rabbit hole of including header files.
On the other hand, STM Cube IDE works similarly to a local C-program. 
If the header files are in the include folder or installed locally, they can use them without much effort.

The STM Cude IDE also supports code generation from Cube MX, letting the user choose which peripherals to be included and adding initiation code before developing a project.
Cube MX also supports importing supported libraries, generating the required include folders and header files for the user beforehand. 
When using the Mbed online compiler, all the peripherals, sensors, and libraries have to be initialized by the user.
It will result in a trade-off between ease of use, performance, and customizability.

One disadvantage of using the STM cube IDE is that all applications will become platform-dependent.
STM cube IDE requires the user to have a Windows computer. 
Mbed online compiler, on the other hand, builds everything in the browser.
When the build finishes, the Mbed online compiler prompts the user to download a .bin file that can be dragged to the STM32 B-L4S5I-IOT01A board, as if storing a file on a flash drive.
The STM32 B-L4S5I-IOT01A board compiles the .bin file and outputs the result to the user, no need to install Cmake or other toolchains. 

Another advantage of using STM cube IDE is debugging. 
STM cube IDE supports debugging with breakpoints, allowing the user to inspect registers when developing.
Mbed online compiler, on the other hand, has no support for debugging, making bug fixing tedious.

\section{Provider arcitechture}
The implementation in this thesis follows the publish/subscribe model, as described in the related work section. 
The STM32 B-L4S5I-IOT01A board is the publisher, and the remote computer will act as the subscriber.
The board has to find where to send the temperature data and sends it in predefined intervals after that. 
The implementation made in this thesis lacks the functionality of acting as a server, serving a request from a consumer.


In the Eclipse Arrowhead system, a provider system will typically be passive, reacting to requests.
When speaking of it in an HTTP sense, the provider will act as a server.
A consumer will usually act as a client and find the provider's address to request services from the provider. 
In this implementation, the provider will find the consumer's address and send the temperature information.
The service it provides is sending the reading from the temperature sensor in specific time intervals. 
It is important not to get hanged upon the terminology as both a provider and a consumer are considered systems in the Eclipse Arrowhead framework. 
In essence, they are the same thing, and we can use the terms interchangeably; it all depends on how the service is defined.
The Eclipse Arrowhead framework uses the term prosumer, a combination of the words provider and consumer, mitigating the confusion caused by these terms.

Depending on the application, it might be more efficient to send the data when the consumer demands it. 
It is possible to imagine the opposite as well.
If a system relies on past information, it might be more efficient to send the data at a specified rate.
If the system has to process the data and perhaps depends on a lot of data, it may be more efficient to use the subscribe/publish model. 
Negating the need for the embedded device to respond to requests can increase energy efficiency by having it boot at specific times to send the data.
If a system relies on real-time data creating a REST API to handle these requests would be the next logical step for developing this example.
The future work section in the upcoming chapter covers implementing a REST API in more detail. 

\section{Comparing different frameworks}
The previous chapters showed that it was possible to use the Eclipse Arrowhead framework with embedded devices. It also showed the advantages of using the framework.
The main advantage was the response time, having on average 17.5 times faster response time than its competitors. 
A faster response time could be a great advantage when dealing with real-time applications when an immediate response is as essential as a correct one. 

One disadvantage of using the Eclipse Arrowhead framework could be the lack of supported hardware.
This thesis was the first example to show that it was possible to connect to the arrowhead framework on embedded devices.
In contrast, Amazon Web Services have many examples of using different hardware. 
Amazon Web Services also has examples showcasing the different sensors and connection possibilities of their supported boards.
Creating an open-source example of using the Eclipse Arrowhead framework could inspire others to start developing embedded Arrowhead applications, expanding the functionality of this implementation.

The eclipse Arrowhead framework requires hardware to run on, a service the customer has to pay for when using Amazon Web Services. 
It is probably cheaper to purchase hardware and run the Eclipse Arrowhead framework on a larger scale, but that requires an upfront investment.
In smaller-scale applications or when trying out a proof of concept, the prices offered by Amazon Web Services are hard to beat.
When using Amazon Web Services, the user relies on that their hardware and network runs continuously.

One significant advantage of using Amazon Web Services is its services offered out of the box.
Lambda functions, S3 buckets, and all the functionality that follows with them is something the user will have to implement on their own if choosing the Eclipse Arrowhead framework.
On the other hand, if the user wants to do something not supported by Amazon Web Services, they will be constrained.
As with the choice of IDE, it will result in a trade-off between ease of use and customizability.

Another significant advantage of Amazon Web Services is the support for and enforcement of using HTTPS.
When deploying a 'Thing' to the AWS IoT Core, it generates a certificate, and the user has to include it in their source code.
The security issues raised in the related work section illustrate the need for HTTPS and other security measures for IoT devices in production.
When developing a proof of concept, as this thesis has been doing, the requirement of HTTPS might be overstated.
It becomes a trade-off between usability, development time, and security. 
If the intent is to demonstrate that the Eclipse Arrowhead framework works on embedded devices, HTTPS and all the extra steps for implementing it might do more harm than good. 

The Eclipse Arrowhead framework supports HTTPS and encourages its developers to use it.
There are guides for generating Arrowhead compliant certificates and support for it in the Java implementation of a client.
However, it proved to be challenging to implement and beyond the scope of this thesis. 
Implementing HTTPS will be covered more in the future work section in the next chapter.