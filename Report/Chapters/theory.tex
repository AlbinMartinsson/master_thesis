%INTRO
The theory behind the implementation in this thesis consists of performing correct SQL-queries to appropriate tables in the Eclipse Arrowhead framework database with the help of Swagger UI REST API.
%AHF
\section{Eclipse Arrowhead framework} 

%Mandatory core systems.
As mentioned in the related work section the Eclipse Arrowhead framework, consists of three mandatory core systems according to Delsing et. al
\begin{itemize}
    \item Service registry system.
    \item Authorization system. 
    \item Orchestration system.\cite{Delsing2017}
\end{itemize} 
%Service registry system.
\subsection{Service registry system}
The service registry system is responsibly for enabling discovery and registring services Delsing et. al. states. 
According to the Eclipse Arrowhead projects own GitHub page the service registry system provides the database which stores the offered services in the local cloud.\cite{Github2021}
The Github page also states the three main objectives of the service registry system are:
\begin{itemize}
    \item To allow the application system to register available services to the database. 
    \item Remove or update available services from the database.
    \item Allow application system to use the lookup functionality of the registry.
\end{itemize}
%Authorization system.
\subsection{Authorization system}
The Authorization system contains two databases for keeping track of which system can consume services from which other systems, depending on whether or not the Application system are in the same cloud or not according to the projects Github page.
The GitHub documentation also states that if the authorization happens within the same cloud it is called intra-cloud authorization and if it happens across two local clouds it is called inter-cloud authorization.\cite{Github2021}
\subsection{Orchestrator system}
%Orchestrator system.
The Orchestration system is responsible for pairing and finding service providers and consumers Delsing et. al. declares.
Delsing et. al. continues to state that the orchestrator also stores the orchestration requirements and the resulting orchestration rules.\cite{Delsing2017} 
The project's documentation argues that the main objective of the orchestrator system is to find an appropriate provider for the requesting consumer system.\cite{Github2021}

%Two types of orchestration
The documentation also states that there are two types of orchestration, store orchestration, and dynamic orchestration.
Store orchestration uses the database orchestration store to find predefined orchestration rules.
Dynamic orchestration on the other hand searches the entire local cloud, or even other clouds, to find the matching provider.\cite{Github2021}
%AFHDB
\section{Arrowhead database}
%Intro
The Eclipse Arrowhead framework can be viewed as a series of database tables that has to be linked with each other.
%Clarification of tables.
One table relevant to all the core systems is the service\_interface table.
It correlates a connection interface, i.e. HTTP-INSECURE-JSON, HTTP-SECURE-JSON and HTTPS-SECURE-JSON, to a specific ID used later on.
\subsection{Service registry system tables}
The tables in the Arrowhead database relevant to the service registry system are
\begin{itemize}
    \item system\_, keeps information about consumers and providers. 
    \item service\_registry, contains information about the different services.
    \item service\_definition, stores service definition name and ID.
    \item service\_registry\_interface\_connection, correlates a service registry ID to a interface ID.
\end{itemize}
\subsection{Authorization system tables}
The tables in the Arrowhead database relevant to the authorization system are
\begin{itemize}
    \item authorization\_intra\_cloud,  adds authorization rules for the provider, consumer and service definition. Dictates which consumers are allowed to connect to which providers and the services they are allowed to use.
    \item authorization\_intra\_cloud\_interface\_connection,  correlates a intra cloud authorization ID to a interface ID.
\end{itemize}
\subsection{Orchestration system tables}
The tables in the Arrowhead database relevant to the orchestrator system are
\begin{itemize}
    \item orchestrator\_store,  contains orchetrator store entry, with information about which endpoints of provides which the consumer can use.
\end{itemize}

%SWAGGER
\section{Swagger UI}
The Eclipse Arrowhead framework has integrated Swagger UI into their core services. 
This means that the databases can be accessed, and altered with using HTTP methods instead of SQL queries.
Each core system has their own swagger UI page which serves as a visual user friendly REST API. 
A REST API is a way to structure an API and consists of the following main principles according to Wikipedia.
\begin{itemize}
    \item Each resource has their own URI.
    \item  
\end{itemize}



%ARM MBED
\section{ARM Mbed}
To enable the use of the Swagger UI on the STM 32 board needs to be connected to the internet. 


\section{Performance measurement}
Response time is measured to compare the performance of the two different protocols, MQTT and HTPP. 
Due to the vastly different implementation of the two frameworks, AWS and Eclipse Arrowhead framework, the only thing worth measuring is the response time when pinging the different frameworks.


 
